\chapter{Query Processing}

The query processing stage is a critical component of our search engine model. It faces the challenge of efficiently handling 
sequences of terms within a query. In this chapter, we discuss the implementation choices for query processing, presenting two 
distinct algorithms: \textit{Document At A Time (DAAT)} and \textit{Block-Max Maxscore (BMM)}.

\section{DAAT}

The \textit{Document At A Time} algorithm is our initial approach to query processing. This algorithm scans the posting lists of 
query terms in parallel, allowing for conjunctive and disjunctive query processing.

\subsection{Conjunctive Query}

In a conjunctive query, the algorithm identifies documents that contain all the terms in the query. This involves scanning the 
posting lists of each term concurrently and identifying documents that appear in all lists. Despite its simplicity, the DAAT 
conjunctive query is not the most efficient, as it scans all posting lists even when unnecessary.

\subsection{Disjunctive Query}

The DAAT algorithm also supports disjunctive queries, where documents containing any of the query terms are retrieved. 
This is achieved by scanning the posting lists and identifying documents that appear in at least one list. While versatile, 
the efficiency of this approach can be improved.

\section{BMM}

The \textit{Block-Max Maxscore} algorithm is our second implementation for query processing. This algorithm introduces new 
data structures saved on the disk to enhance efficiency.

\subsection{Algorithm Overview}

The BMM algorithm employs a block-max scoring strategy. It processes blocks of documents at a time, identifying the maximum 
score within each block. By utilizing this approach, unnecessary scanning of all posting lists is minimized, leading to 
improved efficiency.

\subsection{New Data Structures}

To optimize disk access and retrieval, we have introduced new data structures for storing intermediate results. These 
structures enhance the speed of query processing by minimizing the need for repeated disk reads.
% @todo: add data structures

\section{Implementation Details}

% PARAMETERS FOR BM25 PLEASE
% SCORING FUNCTION PLEASE

% @todo: add implementation details

This chapter provides an overview of the query processing algorithms, shedding light on their strengths, weaknesses, and 
implementation details. The next chapter will delve into performance evaluations and results obtained through experimentation 
with our search engine model.
